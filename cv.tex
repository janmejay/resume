\documentclass[margin,line]{res}

\usepackage{hyperref}
\oddsidemargin -.5in
\evensidemargin -.5in
\textwidth=6.0in
\itemsep=0in
\parsep=0in
% if using pdflatex:
%\setlength{\pdfpagewidth}{\paperwidth}
%\setlength{\pdfpageheight}{\paperheight} 

\newenvironment{list1}{
  \begin{list}{\ding{113}}{%
      \setlength{\itemsep}{0in}
      \setlength{\parsep}{0in} \setlength{\parskip}{0in}
      \setlength{\topsep}{0in} \setlength{\partopsep}{0in} 
      \setlength{\leftmargin}{0.17in}}}{\end{list}}
\newenvironment{list2}{
  \begin{list}{$\bullet$}{%
      \setlength{\itemsep}{0in}
      \setlength{\parsep}{0in} \setlength{\parskip}{0in}
      \setlength{\topsep}{0in} \setlength{\partopsep}{0in} 
      \setlength{\leftmargin}{0.2in}}}{\end{list}}


\begin{document}

\name{Janmejay Singh \vspace*{.1in}}

\begin{resume}
  \section{\sc Contact Information}
  \vspace{.05in}
  \begin{tabular}{@{}p{3.3in}p{3.3in}}
    {\it Phone:}   +91 99726 77488                                                   &  {\it Github:}  \url{https://github.com/janmejay} \\     
    {\it E-mail:}  \href{mailto:singh.janmejay@gmail.com}{singh.janmejay@gmail.com}  &  {\it GPG key:} \href{http://pgp.mit.edu:11371/pks/lookup?op=vindex&search=0x3A9F2343DA21772D}{2048R/DA21772D}\\
  \end{tabular}

\section{\sc Summary}
\vspace{.6cm}
\begin{list2}
\item Senior architect with consistent record of conceptualizing, planning, building and delivering large and complex on-prem and SaaS products.
  \vspace{.1cm}
\item Unique, deep mix of expertise across {\it large scale systems}, {\it systems engineering}, {\it data-center networking}, {\it application development and lifecycle management} and {\it performance engineering}.
  \vspace{.1cm}
\item Someone people love to work with and bank upon.
  \vspace{.1cm}
\item Open-source \& free-software contributor and enthusiast.
\end{list2}

\section{\sc Specialties}
\vspace{.4cm}
\begin{tabular}{@{}p{3.3in}p{3.3in}}
{\bf S}ystem \& Application Architectures for Scale          &  {\bf N}etworking \& Protocol Design                        \\ [0.2cm]
{\bf H}igh-throughput, Multi-tennant SaaS Systems            &  {\bf S}ystem/Application Performance Management            \\ [0.2cm]
{\bf C}, C++, Java/JVM and scripting                         &  {\bf D}istributed Systems                                  \\ [0.2cm]
{\bf E}fficient, Highly-Concurrent Systems                   &  {\bf D}evelopment \& Delivery Best Practices               \\ [0.2cm]
{\bf L}inux/Unix-like Operating Systems                      &  {\bf B}uild-tools/envs, Releases \& Deployments
\end{tabular}

\section{\sc Professional Experience}
\vspace{.4cm}
Acquired close to {\bf 11 years} of {\bf total professional experience} across 3 organizations.\\
\\
{\underline {\bf Flipkart.com}}\\
\\
\href{https://www.flipkart.com/}{\bf Flipkart.com}: Leading Indian E-Commerce Service Provider\hfill {\bf Aug, 2012 - Present}\\
\vspace{-.2cm}\\
Flipkart is the leader in e-commerce segment in India. In addition to maturing the market and establishing service-quality standards, Flipkart has been instrumental in driving a lot of India specific e-commerce innovations such as Cash-on-Delivery mode of payment, Exchange, BigBillionDays etc.\\
\\
Flipkart is built as an eco-system of largely Java/JVM-based services that are deployed over a private cloud and supported by a bunch of horizontal and vertical services together called {\em Flipkart Cloud Platform (FCP)}.\\
\\
As a {\bf Senior Architect}, {\em January, 2017 onwards}, I have been responsible for,\\
\begin{list2}
\item Guiding ground-up re-architecture of extremely critical components like Flipkart\'s transactional-ordered-message-bus providing 4 - 5 years gurantee of seemless scalability without major code-changes
\item Architecting and delivering \href{https://github.com/Flipkart/fk-prof}{\em Flipkart Continious Profiler, fk-prof}, which would be performance management backbone for all Java/JVM based applications in prepration for critical, load-intensive events like BigBillionDay sale. Fk-prof would seemlessly stitch system-level, application-level and JVM-level performance pivots and provide 1-touch performance-profiling capability and visibility at an application-cluster level
\item Responsible for all architectural decisions across large portion of Platform-as-a-Service (PaaS) side of infrastructure team. This includes decisions around introducing new architectural components, addressing scalability challenges, identifying potential issues and reviewing/improving system-design and sometimes low-level design etc.
\item Consultant/Advisor to teams outside of infrastructure group on complex problems such as state-maintainence in low-latency, high-traffic, fault-tolerant critical business systems with competing consistency and availability expectations.
\end{list2}

As an {\bf Architect}, {\em December, 2013 - December, 2016}, I have been responsible for,\\
\begin{list2}
\item Conceptualization, development, benchmarking and rollout of 2 generations of log-indexing-and-storage service called {\em LogSvc}. The 2nd generation of LogSvc (currently in service) handles extreme scale of {\em 2.2 million events per second of indexing} and {\em 1.5 million events per second of storage} traffic and is one of the largest known deployment of certain technologies in the world.
\item Major contribution to open-source projects identified as critical LogSvc dependencies. Eg. {\em Rsyslog} got major features such as \href{https://github.com/rsyslog/rsyslog/pull/614}{\em dynamic-stats} and \href{https://github.com/rsyslog/rsyslog/pull/578}{\em lookup-table} and major performance-fixes such as \href{https://github.com/rsyslog/rsyslog/pull/567}{\em ptcp-work-allocation-improvements} and many other patches. Hadoop, Solr, librdkafka etc were other projects where critical problems were fixed or major improvements were made.
\item write about debugging (pkt-loss, ekart unresponsiveness, cosmos-issues, ad-platform issues etc)
\end{list2}

As an {\bf SDE-3}, {\em August, 2012 - November, 2013}, I have been responsible for,\\
\begin{list2}
\item fill pricing stuff
\end{list2}

{\underline {\bf ThoughtWorks Studios}}\\
\\
\href{https://www.thoughtworks.com/go/}{\bf Go-CD}: Agile release management suite\hfill {\bf August, 2009 - July, 2012}\\
\vspace{-.2cm}\\
Go-CD was the first CI product to bridge development and deployment together, hence introducing the concept of CD (continious deployment), unlocking automation and productivity, in addition to visibility into software moving from development to deployment/release. In summer of 2012, Go-CD was profitable with about 100 customers globally.\\
\\
Heavily multi-threaded, background processing intensive application with JRuby on Rails front end, Spring IOC based Java backend, Spring MVC for parts of UI and RMI endpoints, Hibernate and IBatis talking to an embedded H2 Database, JQuery and Prototype on UI, embedded servlet container Jetty for hosting and Felix OSGi container for plugins.\\
\\
As a {\bf Dev-2 (Senior Developer)}, I was key contributor in first two years and a pseudo tech-lead in the last year, delivering the most critical and challenging of features, keeping performance up to the mark and mentoring other developers on the team technically. My key contributions were,\\
\begin{list2}
\item Designed and implemented several key features while continuously refactoring and improving a 5 yr old codebase for maintainablity, extensiblity and performance
\item Ensured jerk-free continuous delivery, 2-4 weeks production releases(public GA) of the product
\item Contribute to public-domain libraries Go-CD uses as bug or functional/performance fixes
\item Formalizing game changing ideas that made differentiating features keeping us on leading edge
\end{list2}

{\underline {\bf ThoughtWorks}}\\
\\
{\bf Connect}: app platform for CXO level corporate networking\hfill {\bf March, 2008 - July, 2009}\\
\vspace{-.2cm}\\
This platform codenamed Connect was built for a US based consulting giant. It was built by ThoughtWorks team and was later handed over to client engineers for maintenance and use as a platform.\\
\\
Ruby on Rails application backed by MySQL DB using Phusion Passenger and Apache for deployment.\\
\\
As a {\bf Dev-1}, 
\begin{list2}
\item Wrote some key components like user-profile analysis, connection suggestions, community administration, profanity filter etc.
\item Performance tuned database queries, optimized page-load times and tuned content for better user-perceived performance
\item Deployment and environment management configuration-management with puppet/chef became mainstream, standardized build across client organization by building a library over capistrano
\item Trained the support team on handling application deployment for Connect and other projects
\item Managed client expectations on critical and technically challenging features
\end{list2}

{\bf Rezzline}: a hotel booking and search service\hfill {\bf August, 2007 - February, 2008}\\
\\
Rezzline was a hotel search and reservation service based on a unique concept of directly integrating with hoteliers purely as a search engine and was aimed at changing the experience for travelers.\\
\\
Ruby on Rails application backed by MySQL DB using multiple Mongrel instances and Apache as reverse-proxy for deployment. The datastore was filled using data-crawlers operating in the backend.\\
\\
As a {\bf Dev-1}, I co-owned the frontend, backend services, crawler and client-kits to help hoteliers integrate with Rezzline and my major contributions were,\\
\begin{list2}
\item Parallel crawler, relevance calculation algorithm and search mechanism
\item Client kits in Python, Perl, Ruby and Java and providing it to hoteliers running their websites on different tech-stacks
\item Adding features to and maintaining the webapp code and fixing security and performance of the webapp
\item Maintaining the CI, functional testing  and deployment environment
\end{list2}

{\underline {\bf Tavant Technologies}}\\
\\
\href{http://www.tavant.com/warranty-management}{\bf Warranty Management System}: Warranty and inventory management product\hfill {\bf June, 2006 - July, 2007}\\
\vspace{-.2cm}\\
The application was a potential game-changer warranty and extended warranty management system. It was packed with rule engine to allow easy customization at customer's end.\\
\\
Warranty was a Struts 2 MVC(then WebWork) application with Drools based workflow and event processing layer. It was backed by Spring IOC with Hibernate ORM working atop MySQL database. It had dojo based heavily ajax driven UI. It was deployed over embedded Jetty or JBoss AS.\\
\\
As a {\bf Software Engineer},
\begin{list2}
\item Created thin-client UI widget library with tree-table, tabination-framework, spreadsheet like data-grid with filtering and search capabilities, rule-engine UI etc with complex interaction patterns
\item Created an in-browser mini MVC framework to manage widget/component lib functionality cleanly
\item Extracted easy to use interfaces from such common components and created a reusable Struts-2 jsp tag-library based on freemarker templates to enable developers to write purely declarative zero-javascript UI and helped several teams across the organization build bug-free UI with it
\end{list2}

\section{\sc Community contribution}
        {\underline {\bf Open Source contribution}}\\
        \vspace{-.2cm}\\
        I firmly believe in Free and Open Source philosophy and do my best to contribute back to the (F)OSS community in personal time.

        I am co-creator and maintainer of popular test parallelization solution called \href{http://test-load-balancer.github.com}{Test Load Balancer} (released under BSD 2-clause license) used by several projects within ThoughtWorks and outside of it. 
        TLB supports several test-runner and build-tool combinations across programming languages and provides one stop, minimal configuration solution to all test-parallelization needs. 
        It has also been the driver for some really interesting work we did, for instance, coming up with a novel approach to set-partitioning using genetic algorithm.

        In addition to my personal projects, I have also contributed patches amounting to several feature enhancements, additions and functional or performance bug-fixes to multiple open-source projects, some of the more popular being JRuby, Maven Surefire, Objenesis, UrlRewrite Filter etc.

        I also develop and maintain other tiny projects on github like .emacs.d repository that several emacs users collaborate on and a little android app meant to teach colors to kids etc.

        While working for Tavant, I drove the \href{http://sourceforge.net/projects/infrared}{InfraRED} (a J2EE performance monitoring app) integration effort to plug InfraRED with JBoss, Weblogic, Websphere, Jetty, Resin, Tomcat and OC4J etc.

        {\underline {\bf Conferences}}\\
        \vspace{-.2cm}\\
        Being a strong believer in sharing of knowledge or ideas and hacker ethics, I have talked at several conferences about the projects that I work on and problems we solve including some of the more well known conferences like {\em JavaOne}, {\em RubyConf}, {\em Great Indian Developer Summit} and lesser known {\em Devops days}, {\em BarCamp}, {\em Rootconf} and {\em XConf} etc.\\

\section{\sc Formal \\Education}

{\bf Indian Institute of Science, Bangalore}, Karnataka, India\\
\vspace{-.3cm}
\begin{list1}
\item[] Short-term course: Artificial Intelligence and Intelligent Agents, 2010
\end{list1}

{\bf Indian Institute of Technology, Roorkee}, Uttarakhand, India\\
\vspace{-.3cm}
\begin{list1}
\item[] B.Tech, 2006
\end{list1}

{\bf Kendriya Vidyalaya, Hisar}, Haryana, India\\
\vspace{-.3cm}
\begin{list1}
\item[] 12th Std. CBSE, 2001
\item[] 10th Std. CBSE, 1999
\end{list1}


\end{resume}
\end{document}

